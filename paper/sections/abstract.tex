\begin{abstract}
Scientific knowledge graphs (KGs) encode entities and typed relations across
domains such as physics, astronomy, and materials science, yet they remain
incomplete: missing edges and entities limit downstream reasoning.  We introduce
\emph{Harmony}, a framework that treats theory discovery as the search for KG
mutations---new edges or entities---that maximise a composite quality metric.
The \emph{Harmony score} combines four complementary signals: \textbf{compressibility}
(minimum description length proxy), \textbf{coherence} (path-semantic consistency),
\textbf{symmetry} (entity-type behavioural uniformity via Jensen--Shannon divergence),
and \textbf{generativity} (link-prediction learnability via DistMult).  An LLM
proposer generates candidate theory-level propositions, which are validated,
scored, and archived in a MAP-Elites quality-diversity grid.  Four islands with
distinct strategies---refinement, combination, and novelty---explore the
proposal space concurrently, with periodic migration.  Calibration experiments on
linear algebra and periodic table KGs show Harmony scores 31--65\% above
frequency baselines.  On three discovery domains (astronomy, physics, materials
science), the system produces valid, diverse proposals that improve Hits@10 over
a standalone DistMult baseline.  Expert rubric evaluation confirms that top
proposals achieve plausibility scores $\geq 3.0$ on a 5-point scale.
\end{abstract}
