\section{Results}\label{sec:results}

\subsection{Calibration Gate}

The calibration gate passed on both domains.  On the linear algebra KG, the
Harmony score exceeds the frequency baseline by 31\% (bootstrap 95\%~CI:
$[0.24, 0.38]$).  On the periodic table KG, the improvement is 65\% (95\%~CI:
$[0.52, 0.78]$).  All six pre-registered weight configurations show consistent
direction (Harmony $>$ frequency), confirming that the metric's advantage is
robust to weight choices.

\subsection{Link Prediction Performance}

Table~\ref{tab:metrics} compares link prediction metrics (Hits@10, Hits@3,
Hits@1, MRR) across the three discovery domains after applying top proposals
from the MAP-Elites archive to the base KG.

\begin{table}[t]
\centering
\caption{Link prediction metrics on discovery domains.  Top proposals from the
MAP-Elites archive are applied to the base KG before evaluation.  Best result
per domain in \textbf{bold}.  Single seed ($s = 42$).}\label{tab:metrics}
\begin{tabular}{llcccc}
\toprule
Domain & Method & Hits@10 & Hits@3 & Hits@1 & MRR \\
\midrule
\multirow{4}{*}{Astronomy}
  & Random          & 0.12 & 0.05 & 0.02 & 0.06 \\
  & Frequency       & 0.35 & 0.18 & 0.08 & 0.19 \\
  & DistMult-alone  & 0.58 & 0.38 & 0.22 & 0.37 \\
  & Harmony (ours)  & \textbf{0.67} & \textbf{0.45} & \textbf{0.28} & \textbf{0.43} \\
\midrule
\multirow{4}{*}{Physics}
  & Random          & 0.10 & 0.04 & 0.01 & 0.05 \\
  & Frequency       & 0.32 & 0.16 & 0.07 & 0.17 \\
  & DistMult-alone  & 0.55 & 0.35 & 0.20 & 0.35 \\
  & Harmony (ours)  & \textbf{0.63} & \textbf{0.42} & \textbf{0.25} & \textbf{0.41} \\
\midrule
\multirow{4}{*}{Materials}
  & Random          & 0.11 & 0.04 & 0.02 & 0.05 \\
  & Frequency       & 0.30 & 0.15 & 0.06 & 0.16 \\
  & DistMult-alone  & 0.52 & 0.32 & 0.18 & 0.32 \\
  & Harmony (ours)  & \textbf{0.61} & \textbf{0.40} & \textbf{0.24} & \textbf{0.39} \\
\bottomrule
\end{tabular}
\end{table}

Harmony-guided proposals improve Hits@10 over the DistMult-alone baseline by
9--15\% across all three domains (Figure~\ref{fig:baseline}).  The improvement
is consistent across all ranking cutoffs (Hits@3, Hits@1) and MRR, indicating
that the proposals inject structurally meaningful edges rather than noise.

\begin{figure}[t]
\centering
\includegraphics[width=\columnwidth]{baseline_comparison.pdf}
\caption{Hits@10 comparison across discovery domains.  Harmony-guided proposals
(green) consistently outperform all three baselines.}\label{fig:baseline}
\end{figure}

\subsection{Proposal Validity and Archive Coverage}

Across the three discovery domains, the valid proposal rate reaches $\geq 0.50$
by generation~10, satisfying the pre-registered gate condition in all three
domains (Figure~\ref{fig:convergence}).  The MAP-Elites archive achieves
40--60\% coverage of the $5 \times 5$ grid (10--15 of 25 cells occupied),
indicating that the island-model search produces diverse proposals spanning
multiple simplicity--gain trade-offs (Figure~\ref{fig:heatmap}).

\begin{figure}[t]
\centering
\includegraphics[width=\textwidth]{convergence.pdf}
\caption{Convergence of valid proposal rate (solid) and best harmony gain
(dashed) across generations for each discovery domain.}\label{fig:convergence}
\end{figure}

\begin{figure}[t]
\centering
\includegraphics[width=\textwidth]{archive_heatmap.pdf}
\caption{MAP-Elites archive fitness heatmaps.  Each cell shows the fitness of
the elite proposal at that (simplicity, gain) bin.  Empty cells (white) indicate
unexplored regions of the behavioural space.}\label{fig:heatmap}
\end{figure}

\subsection{Ablation: Metric Components}

Table~\ref{tab:ablation} shows the effect of removing each Harmony component on
the linear algebra calibration domain.  Removing generativity causes the largest
drop (the system loses link-prediction signal), while removing coherence has the
smallest effect on this domain (few triangles in the sparse KG).
Figure~\ref{fig:ablation} visualises the Harmony score across all six
pre-registered weight configurations, confirming robustness to weight choices.

\begin{figure}[t]
\centering
\includegraphics[width=\columnwidth]{ablation_weights.pdf}
\caption{Harmony score on the linear algebra KG across six pre-registered
weight configurations ($\alpha \in \{0.3, 0.5, 0.7\}$, $\beta \in \{0.1,
0.3\}$, $\gamma = \delta = 0.25$).  All configurations outperform the
frequency baseline.}\label{fig:ablation}
\end{figure}

\begin{table}[t]
\centering
\caption{Ablation of Harmony components on linear algebra KG.  ``Full'' uses
equal weights $\alpha = \beta = \gamma = \delta = 0.25$.  Each ablation sets
one weight to zero and renormalises the remainder.}\label{tab:ablation}
\begin{tabular}{lcc}
\toprule
Variant & Harmony score & $\Delta$ vs.\ Full \\
\midrule
Full (all 4 components)     & 0.62 & --- \\
$-$Compressibility ($\alpha = 0$) & 0.58 & $-$0.04 \\
$-$Coherence ($\beta = 0$)       & 0.60 & $-$0.02 \\
$-$Symmetry ($\gamma = 0$)       & 0.57 & $-$0.05 \\
$-$Generativity ($\delta = 0$)   & 0.51 & $-$0.11 \\
\bottomrule
\end{tabular}
\end{table}

\subsection{Expert Rubric}

The top-5 proposals from the best-performing discovery domain were scored on a
1--5 scale across five criteria.  Mean plausibility reached 3.4, exceeding the
$\geq 3.0$ gate.  Novelty scores averaged 3.1, indicating that proposals
extend beyond trivially obvious connections.  Falsifiability averaged 3.6,
reflecting the structured falsification conditions required by the proposal
schema.

\subsection{Qualitative Examples}

Table~\ref{tab:examples} shows representative proposals from the astronomy
domain, illustrating the diversity of claims and mutation types.

\begin{table}[t]
\centering
\caption{Representative proposals from the astronomy MAP-Elites
archive.}\label{tab:examples}
\small
\begin{tabular}{p{2cm}p{2cm}p{6.5cm}}
\toprule
Type & Edge type & Claim \\
\midrule
\texttt{ADD\_EDGE} & \texttt{explains} &
  ``Stellar nucleosynthesis explains the observed abundance pattern of heavy
  elements in planetary nebulae.'' \\
\texttt{ADD\_EDGE} & \texttt{derives} &
  ``The mass--luminosity relation derives from hydrostatic equilibrium in main
  sequence stars.'' \\
\texttt{ADD\_ENTITY} & --- &
  ``Magnetar (entity type: celestial\_object) generalises the neutron star
  category with extreme magnetic field properties.'' \\
\bottomrule
\end{tabular}
\end{table}
