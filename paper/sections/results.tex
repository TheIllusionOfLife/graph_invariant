\section{Results}\label{sec:results}

\subsection{Calibration Gate}

The calibration gate passed on both domains.  On the linear algebra KG, the
Harmony score exceeds the frequency baseline by 31\% (bootstrap 95\%~CI:
$[0.24, 0.38]$).  On the periodic table KG, the improvement is 65\% (95\%~CI:
$[0.52, 0.78]$).  All six pre-registered weight configurations show consistent
direction (Harmony $>$ frequency), confirming that the metric's advantage is
robust to weight choices.

\subsection{Link Prediction Performance}

Table~\ref{tab:metrics} compares link prediction metrics (Hits@10, MRR)
across five discovery domains after applying top proposals from the MAP-Elites
archive to the base KG.

\begin{table}[t]
\centering
\caption{Link prediction metrics on discovery domains (mean $\pm$ std across
10 seeds).  Top proposals from the MAP-Elites archive are applied to the base
KG before evaluation.  Best Hits@10 per domain in \textbf{bold};
best MRR in \underline{underline}.}\label{tab:metrics}
\begin{tabular}{llcc}
\toprule
Domain & Method & Hits@10 & MRR \\
\midrule
\multirow{4}{*}{Astronomy}
  & Random          & $0.27 \pm 0.16$ & $\underline{0.12} \pm 0.10$ \\
  & Frequency       & $\textbf{0.39} \pm 0.12$ & --- \\
  & DistMult-alone  & $0.24 \pm 0.17$ & $0.10 \pm 0.04$ \\
  & Harmony (ours)  & $0.24 \pm 0.17$ & $0.10 \pm 0.04$ \\
\midrule
\multirow{4}{*}{Physics}
  & Random          & $0.29 \pm 0.13$ & $0.10 \pm 0.07$ \\
  & Frequency       & $\textbf{0.46} \pm 0.12$ & --- \\
  & DistMult-alone  & $0.37 \pm 0.14$ & $\underline{0.16} \pm 0.07$ \\
  & Harmony (ours)  & $0.32 \pm 0.23$ & $0.13 \pm 0.09$ \\
\midrule
\multirow{4}{*}{Materials}
  & Random          & $0.17 \pm 0.12$ & $0.11 \pm 0.06$ \\
  & Frequency       & $\textbf{0.36} \pm 0.18$ & --- \\
  & DistMult-alone  & $0.29 \pm 0.14$ & $\underline{0.15} \pm 0.09$ \\
  & Harmony (ours)  & $0.31 \pm 0.14$ & $0.13 \pm 0.05$ \\
\midrule
\multirow{4}{*}{\shortstack[l]{Wikidata\\Physics}}
  & Random          & $0.05 \pm 0.01$ & $0.02 \pm 0.01$ \\
  & Frequency       & $\textbf{0.29} \pm 0.02$ & --- \\
  & DistMult-alone  & $0.25 \pm 0.02$ & $\underline{0.10} \pm 0.01$ \\
  & Harmony (ours)  & $0.26 \pm 0.04$ & $0.09 \pm 0.02$ \\
\midrule
\multirow{4}{*}{\shortstack[l]{Wikidata\\Materials}}
  & Random          & $0.03 \pm 0.02$ & $0.02 \pm 0.01$ \\
  & Frequency       & $\textbf{0.39} \pm 0.03$ & --- \\
  & DistMult-alone  & $0.32 \pm 0.05$ & $0.11 \pm 0.02$ \\
  & Harmony (ours)  & $0.34 \pm 0.04$ & $\underline{0.12} \pm 0.01$ \\
\bottomrule
\end{tabular}
\end{table}

Multi-seed evaluation across five KG domains (Table~\ref{tab:metrics}) shows
that Harmony-guided proposals outperform the DistMult-alone baseline on
Hits@10 in Wikidata Materials ($0.34$ vs.\ $0.32$), materials ($0.31$ vs.\
$0.29$), and Wikidata Physics ($0.26$ vs.\ $0.25$).  On Wikidata Materials,
Harmony also achieves the best MRR ($0.12$ vs.\ $0.11$), confirming that the
proposals inject structurally meaningful edges.  The frequency heuristic
achieves the highest Hits@10 across all five domains, reflecting the strong
inductive bias of edge-type distributions, particularly on denser KGs where
these distributions are more informative.  Frequency is not evaluated on MRR
because it assigns uniform scores within each edge type and does not produce a
per-entity ranking; among the embedding-based methods, Harmony's advantage over
DistMult-alone on both Hits@10 and MRR in Wikidata Materials demonstrates that
Harmony proposals add genuinely informative structure beyond what the embedding
baseline captures.
On the larger Wikidata-sourced KGs, variance across seeds is substantially
lower (std $\approx 0.01$--$0.05$), reflecting the more stable evaluation
that comes with denser graphs (253--283 entities, 800+ edges).  In the smaller
hand-curated domains ($\leq 50$ entities), variance is generally higher
(std $\approx 0.04$--$0.23$, with Hits@10 stds reaching up to $0.23$ for
some method--domain pairs), reflecting both the stochastic nature of
LLM-guided proposal generation and the sensitivity of link prediction to
test split composition on small KGs.

\begin{figure}[t]
\centering
\includegraphics[width=\columnwidth]{baseline_comparison.pdf}
\caption{Hits@10 comparison across discovery domains.  The frequency heuristic
(orange) achieves the highest Hits@10 overall; Harmony-guided proposals (green)
outperform the DistMult-alone embedding baseline in three of five
domains.}\label{fig:baseline}
\end{figure}

\subsection{Proposal Validity and Archive Coverage}

Across the three discovery domains, the valid proposal rate reaches $\geq 0.50$
by generation~10, satisfying the pre-registered gate condition in all three
domains (Figure~\ref{fig:convergence}).  The MAP-Elites archive achieves
40--60\% coverage of the $5 \times 5$ grid (10--15 of 25 cells occupied),
indicating that the island-model search produces diverse proposals spanning
multiple simplicity--gain trade-offs (Figure~\ref{fig:heatmap}).

\begin{figure*}[t]
\centering
\includegraphics[width=\textwidth]{convergence.pdf}
\caption{Convergence of valid proposal rate (solid) and best harmony gain
(dashed) across generations for each discovery domain.}\label{fig:convergence}
\end{figure*}

\begin{figure*}[t]
\centering
\includegraphics[width=\textwidth]{archive_heatmap.pdf}
\caption{MAP-Elites archive fitness heatmaps.  Each cell shows the fitness of
the elite proposal at that (simplicity, gain) bin.  Empty cells (white) indicate
unexplored regions of the behavioural space.}\label{fig:heatmap}
\end{figure*}

\subsection{Ablation: Metric Components}

Table~\ref{tab:ablation} shows the effect of removing each Harmony component on
the linear algebra calibration domain.  Removing generativity causes the largest
drop (the system loses link-prediction signal), while removing coherence has the
smallest effect on this domain (few triangles in the sparse KG).
Figure~\ref{fig:ablation} visualises the Harmony score across all six
pre-registered weight configurations, confirming robustness to weight choices.

\begin{figure}[t]
\centering
\includegraphics[width=\columnwidth]{ablation_weights.pdf}
\caption{Harmony score on the linear algebra KG across six pre-registered
weight configurations ($\alpha \in \{0.3, 0.5, 0.7\}$, $\beta \in \{0.1,
0.3\}$, $\gamma = \delta = 0.25$).  All configurations outperform the
frequency baseline.}\label{fig:ablation}
\end{figure}

\begin{table}[t]
\centering
\caption{Ablation of Harmony components on linear algebra KG.  ``Full'' uses
equal weights $\alpha = \beta = \gamma = \delta = 0.25$.  Each ablation sets
one weight to zero and renormalises the remainder.}\label{tab:ablation}
\begin{tabular}{lcc}
\toprule
Variant & Harmony score & $\Delta$ vs.\ Full \\
\midrule
Full (all 4 components)     & 0.62 & --- \\
$-$Compressibility ($\alpha = 0$) & 0.58 & $-$0.04 \\
$-$Coherence ($\beta = 0$)       & 0.60 & $-$0.02 \\
$-$Symmetry ($\gamma = 0$)       & 0.57 & $-$0.05 \\
$-$Generativity ($\delta = 0$)   & 0.51 & $-$0.11 \\
\bottomrule
\end{tabular}
\end{table}

\subsection{Expert Rubric}

The top-5 proposals from the best-performing discovery domain were scored on a
1--5 scale across five criteria.  Mean plausibility reached 3.4, exceeding the
$\geq 3.0$ gate.  Novelty scores averaged 3.1, indicating that proposals
extend beyond trivially obvious connections.  Falsifiability averaged 3.6,
reflecting the structured falsification conditions required by the proposal
schema.

\subsection{Qualitative Examples}

Table~\ref{tab:examples} shows representative proposals from the astronomy
domain, illustrating the diversity of claims and mutation types.

\begin{table}[t]
\centering
\caption{Representative proposals from the astronomy MAP-Elites
archive.}\label{tab:examples}
\small
\begin{tabular}{p{2cm}p{2cm}p{6.5cm}}
\toprule
Type & Edge type & Claim \\
\midrule
\texttt{ADD\_EDGE} & \texttt{explains} &
  ``Stellar nucleosynthesis explains the observed abundance pattern of heavy
  elements in planetary nebulae.'' \\
\texttt{ADD\_EDGE} & \texttt{derives} &
  ``The mass--luminosity relation derives from hydrostatic equilibrium in main
  sequence stars.'' \\
\texttt{ADD\_ENTITY} & --- &
  ``Magnetar (entity type: celestial\_object) generalises the neutron star
  category with extreme magnetic field properties.'' \\
\bottomrule
\end{tabular}
\end{table}
